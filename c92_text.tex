\documentclass[a5j,openany,twoside]{jsbook}
%\documentclass[9pt,a5paper,tombo,openany]{jsbook}

%\usepackage{graphicx} % support the \includegraphics command and options
\usepackage{colortbl}
\usepackage[dvipdfmx]{graphicx}
\usepackage{wrapfig}
\usepackage{url}
\usepackage{enumitem}
\usepackage[inner=2.5cm,outer=1.5cm,bottom=2.5cm,top=2.5cm]{geometry}


%\setlist[description]{leftmargin=*}
%%% The "real" document content comes below...

%\title{可視化法学}
%\author{芝尾幸一郎}
%\date{} % Activate to display a given date or no date (if empty),
         % otherwise the current date is printed

\begin{document}

%\thispagestyle{empty}
%
%\includegraphics[width=11cm,clip]{img/front.pdf}
%
%\newpage
%\thispagestyle{empty}
% %←白紙ページの為に全角文字列が入ってます。
%
\chapter{可視化法学}

\setcounter{page}{1}

\section{可視化法学とは}

可視化法学とは、とかく分かりにくい法律を情報技術を用いて可視化する試みだ。「法律の可視化」で検索しても出てくるのは取り調べの可視化ばかりだが、可視化法学とはそういう意味ではなく情報工学を用いて法律間のつながりや構造を明らかにするプロジェクトだ。
法律は杓子定規で古臭くコンピュータなどは使わず、法律と情報技術は無縁だと考えられがちだ。
だから、これまで法律に対して可視化するとか情報技術の対象にするという風には考えられなかった。理由は2つ考えられる。
一つには法律は条文の数が多く構造は複雑で昔の非力なコンピュータでは処理が出来なかった点。もう一つは、情報工学と法律学では研究している人たちが異なっているのでなかなか法律学の方に情報技術を活かすという発想が少なかったからだ。

現在ではコンピュータの処理能力も向上し、またビックデータをはじめとしたデータ分析の技術も進化したため複雑な法律の構造も解析出来るようになってきた。

この冊子は法律を情報技術を使って捉え直すプロジェクトの一環として作成した。法律の意外な一面が見えたら嬉しい。

\section{法律≒プログラミングコード}

法律は、英語でCODE(コード)とも呼ばれる。エンジニアがコンピュータでプログラムを動かすためにプログラミング言語を用いて書くものも、CODE(ソースコード)と呼ばれる。同じ名前で呼ばれるように、一見別々だと思われがちな法律もソースコードも似たような特徴を持つ。

その特徴とは、

\begin{enumerate}
 \item 極めて論理的で、構造化された言語で書かれている
 \item 社会やコンピュータなど複雑なシステムを動かすのに使う
\end{enumerate}

といったものだ。
ソースコードは分析と可視化が進んでおり、何回実行されたかや、どの処理が負荷が高いかなどを可視化したり、ソースコードの更新履歴を可視化するなど単にソースコードをコンピュータを動かすために使うだけではなく、ソースコードそれ自体が分析や効率化の対象になっている。
このようなソースコードに対するアプローチを、同じくコードである法律にも適応出来ないかと考える人が出始めている。
これは、プログラマー向けイベント「Developer Summit2015」の基調講演においてドワンゴの川上会長が喋った例だが、ソースコードの複雑さを図る指標を法律に応用した事例を紹介している\footnote{\url{http://weekly.ascii.jp/elem/000/000/306/306175/}}。川上氏いわく、ソースコードの複雑さを測る指標があり、その値が75を超えれば、どのような修正もバグを生むと言われる。その指標で、著作権法 第47条の10を解析すると100を超えたそうだ。
このようにソースコードを分析する技法や、データを分析する手法を使って法律を分析したら法律をもっと理解できるようになるのではないかと考えた。また、将来法律の修正がこのようなツールを駆使して行われるようになったら可視化法学の試みは意味を持つようになるだろう。

\section{なぜ一介のプログラマーが法律に興味を持つのか}

私はプログラマーだ。法律を業務で利用することはない一般的にプログラマーは法律に興味を持たない。業務で利用しないしプログラムは理系の学問で法律は文系の学問であり関係がないと考えられているからだろう。私が法律に興味を持つのはなぜだろう?
一つには前に書いたように、法律もプログラムも同じ(コード=CODE)で出来ているからだ。
もう一つは、情報技術の知見を活かして法律をもっと良いものに出来ないかと思っているからだ。
情報技術の発展によりプログラマーは、自分たちが書くプログラミング言語をより便利に効率的にしている。例えば、毎回変数の型を書くのではなく型推論の仕組みを導入する\footnote{昔のプログラミング言語であれば int a=0;などと書いていたが、それをval a=0と置き換える}など、機械の力をうまく使ってプログラマーはより楽に効率的にプログラムを書けるようにしている。
一方で、法律を書くのにそのような支援ツールを使っているようには見えない。もっと効率的に出来るはずの要素があるのに、それを機械化などを導入せず、人力でどうにかしようとしているように見える。
このような現状をもう少し楽に出来るのではないかと考えている。

\section{可視化法学の仕組み}

可視化法学の仕組みを説明する。巻末に技術的な詳細を載せている。可視化法学は次のように作った。

\begin{enumerate}
 \item 法律の条文を手に入れる
 \item 法律間の参照を抜き出す
 \item コンピュータを使って可視化する
\end{enumerate}

\subsubsection{法律の条文を手に入れる}

法律の分析を行うためには、その元になる法律の条文を手に入れる必要がある。電子版のCD-ROMを売っている\footnote{\url{http://iss.ndl.go.jp/books/R100000002-I027118185-00}}のでそれを使っても良いと思うが、高額だったり絶版だったりと入手が困難なことも多いのでネットにある法律情報から条文を手に入れることにする。

総務省の提供している法令データ提供システムから情報を取得する。

\url{http://law.e-gov.go.jp/}

\subsubsection{法律間の参照を抜き出す}

今回の可視化は、法律に含まれる他の法律への参照をリンクのようにつなげることで行なっている。例えば、救急救命士法には刑法への参照が含まれる。


\begin{wrapfigure}{l}{30mm}
 \begin{center}
  \includegraphics[width=25mm,clip]{img/link-crop.pdf}
  \caption{法律間の関連を可視化}
 \end{center}
\end{wrapfigure}

\begin{quote}

\textbf{救急救命士法}
(秘密保持義務等)
第十七条  指定登録機関の役員若しくは職員又はこれらの職にあった者は、登録事務に関して知り得た秘密を漏らしてはならない。
2  登録事務に従事する指定登録機関の役員又は職員は、\textbf{刑法}(明治四十年法律第四十五号)その他の罰則の適用については、法令により公務に従事する職員とみなす。

\end{quote}

図示すると次のようになる。

救急救命士法は刑法への参照がある\footnote{救急救命士は、刑法の適応においては、公務員とみなすという文で刑法を参照している}ので、刑法と線をつなげている。このような参照関係を調べるためプログラムを書き、全ての法律を調べそのような他の法律への参照をすべて抜き出していった。

\subsubsection{コンピュータを使って可視化する}

Gephi\footnote{\url{http://oss.infoscience.co.jp/gephi/gephi.org/}}というソフトを使って、法律間の可視化を行う。色々な法律から参照される法律は、その参照数に応じて円のサイズを大きくしている。例えば刑法は、刑事分野で450個もの法令から参照されている。その為、刑法は大きな円になっている。
また、「胡一凡」という開発者の名前がついたレイアウトアルゴリズムを用いてグラフを作っている。結果として似た法令は近い場所に集まるようになっている。例えば刑事分野では、売春防止法と少年法、更生保護法、更生保護事業法は近い場所にあり、刑事分野では厚生や福祉に近いジャンルであることが分かる。次章からは、そのような仕組みで作った様々な法律を見ていこう。

\chapter{様々な法律分野の可視化}

前章のルールを適応して幾つかの法律分野を可視化した。分野は、刑事分野、文化分野、河川分野、鉱業分野、教育分野、憲法分野、社会保険分野だ。法律の分野によって相互依存の複雑さが違うので、ごちゃごちゃしていたり、スッキリしていたりする。ざっと見た限りだと、社会保険分野(年金、社会保障)と労働分野(労働基準法、雇用保険)という厚生省管轄の法律はやたらと複雑だ。多分、何か一つの法律を変えるとバグが出ちゃうような感じだろう。

\section{参照した法律の分野}

参照した法律の分野は、総務省の提供している法令データ提供システムに準拠している。法令データ提供システムによると法律の分野は次のようになっているようだ。このうち、今回取り上げた法律は背景を灰色にしている。

\begin{table}[htb]
  \begin{tabular}{|l|l|l|l|l|l|}  \hline
\cellcolor[gray]{0.85} 憲 法 & \cellcolor[gray]{0.85} 刑 事 & 財務通則 & 水産業 & 観 光\\
国 会 & 警 察 & 国有財産 & \cellcolor[gray]{0.85} 鉱 業 & 郵 務\\
行政組織 & 消 防 & 国 税 & 工 業 & 電気通信\\
国家公務員 & 国土開発 & 事 業 & 商 業 & \cellcolor[gray]{0.85} 労 働\\
行政手続 & 土 地 & \cellcolor[gray]{0.85}国 債 & 金融・保険 & 環境保全\\
統 計 & 都市計画 & \cellcolor[gray]{0.85}教 育 & 外国為替・貿易 & 厚 生\\
地方自治 & 道 路 & \cellcolor[gray]{0.85}文 化 & 陸 運 & 社会福祉\\
地方財政 & \cellcolor[gray]{0.85}河 川 & 産業通則 & 海 運 & \cellcolor[gray]{0.85} 社会保険\\
司 法 & 災害対策 & 農 業 & 航 空 & 防 衛\\
民 事 & 建築・住宅 & 林 業 & 貨物運送 & 外 事\\ \hline
  \end{tabular}
\end{table}

\section{刑事分野}

\includegraphics[width=9cm,clip]{img/keiji-crop.pdf}

刑事分野の法律の数は、\textbf{713}本、参照数は、\textbf{1083}個である。

まず参照される法律の順に主要な法律を列挙する。

\begin{table}[htb]
  \begin{tabular}{|l|l|}  \hline
法律名 & 参照された数 \\ \hline \hline
刑法 & 450 \\
刑事訴訟法 & 112 \\
売春防止法 & 54 \\
少年法 & 44 \\
更生保護法 & 33 \\
更生保護事業法 & 29 \\
犯罪収益移転防止法 & 28 \\
刑事収容施設法 & 26 \\
組織犯罪処罰法 & 25 \\
少年院法 & 23 \\ \hline
  \end{tabular}
\end{table}

刑事分野の法律で中心になる法律は刑法だ。450個の法律から参照されている。その他、刑事訴訟法も多く参照される。ただ刑法、刑事訴訟法両方に結びついている法律はそこまで多くない。刑法と結びついた法律は多いが、刑事分野でそれ以外の法律と結びついている法律はそこまで多いわけではない。

前にも書いたが、売春防止法や少年法、更生保護法などは結構近い位置にあり、刑事分野の中でも更生や福祉に力点をおいた法律であるように考えられる。

\begin{wrapfigure}{l}{35mm}
 \begin{center}
  \includegraphics[width=30mm,clip]{img/kousei.pdf}
  \caption{刑事の中の福祉}
 \end{center}
\end{wrapfigure}

また組織犯罪処罰法や犯罪収益移転防止法は近い位置にあり、要は暴力団対策なのだろう。

\section{文化分野}

\includegraphics[width=8cm,clip]{img/bunka-crop.pdf}

文化分野の法律の数は\textbf{303}本、参照数は\textbf{355}個である。

まず参照される法律の順に主要な法律を列挙する。

\begin{table}[htb]
  \begin{tabular}{|l|l|}  \hline
法律名 & 参照された数 \\ \hline \hline
文化財保護法 & 121 \\
祝日法 & 68 \\
著作権法 & 44 \\
気象業務法 & 20 \\
管理事業法 & 13 \\
文化財保護法施行令 & 13 \\
宗教法人法 & 12 \\ \hline
  \end{tabular}
\end{table}

文化分野では大きく分けて3つのカテゴリーがある。「国民の祝日に関する法律」を中心とした分野、「文化財保護法」を中心とした分野、「著作権法」を中心とした分野だ。漫画やアニメなどに興味を持っていると、文化分野ではどうしても著作権の方が大きいような気になるが、実際は文化財保護法の方が関連する法律は多い。建物を建築中に遺跡をうっかり掘り当ててしまうと遺跡の発掘調査のため建築を一時中断せざるをえない。現実の建物が関わるので、文化財保護法のほうが実際は影響がでかいのだろう。

\section{河川分野}

\includegraphics[width=9cm,clip]{img/河川-crop.pdf}

河川分野の法律の数は\textbf{192}本、参照数は\textbf{227}個である。

まず参照される法律の順に主要な法律を列挙する。

\begin{table}[htb]
  \begin{tabular}{|l|l|}  \hline
法律名 & 参照された数 \\ \hline \hline
河川法 & 170 \\
河川法施行令 & 16 \\
特定多目的ダム法 & 14 \\
運河法 & 11 \\
河川管理施設等構造令 & 3 \\ \hline
  \end{tabular}
\end{table}

基本的に河川に関する法律のカテゴリーになる。
河川分野は結構シンプル。河川法が170個ほどの参照がある以外は運河法や、特定多目的ダム法など参照数が10個前後と極端に少なくなる。基本的に河川に関しては河川法を踏まえれば良いということになるだろう。最近はドローンを飛ばしたり畑を作ったりと結構自由に河川を使う人が増えたので、そこらへんはどう変わっていくのか興味がある。

\section{鉱業分野}

\includegraphics[width=9cm,clip]{img/鉱業-crop.pdf}

鉱業分野の法律の数は\textbf{184}本、参照数は\textbf{312}個である。

まず参照される法律の順に主要な法律を列挙する。

\begin{table}[htb]
  \begin{tabular}{|l|l|}  \hline
法律名 & 参照された数 \\ \hline \hline
鉱業法 & 68 \\
鉱山保安法 & 58 \\
石油パイプライン事業法 & 29 \\
採石法 & 24 \\
砂利採取法 & 18 \\
石油の備蓄の確保等に関する法律 & 13 \\
技術研究組合法 & 12 \\
金属鉱業等鉱害対策特別措置法 & 10 \\
鉱山保安法施行規則 & 10 \\ \hline
  \end{tabular}
\end{table}

鉱業分野は主に、石炭などの鉱物資源に関する法律を定めている。昔は石炭だったが石炭から石油にエネルギー源が移り変わるにつれて、石油に関する法律も管轄するようになった。主務官庁は経産省である。面白いのは鉱業法に並んで、鉱山保安法という炭鉱夫などの鉱山作業者の保安監督を行う法律が大きなウエイトを占めていることだ。初めて知ったが鉱山の保安を取り締まるために、警察とは別に鉱務監督官という職種があり、厚生労働省で麻薬の捜査に関わる麻薬取締官のように特別司法警察職員としての地位が与えられているそうだ。確かに考えてみたら鉱山は山奥にあるし、そのまた奥の坑道で事故や犯罪が起こったら、いちいち警察を呼んでいたら間に合わないという事情があるのだろう。

\section{教育分野}


\includegraphics[width=9cm,clip]{img/kyouiku-crop.pdf}

教育分野の法律の数は\textbf{842}本、参照数は\textbf{1503}個である。

まず参照される法律の順に主要な法律を列挙する。


\begin{table}[htb]
  \begin{tabular}{|l|l|}  \hline
法律名 & 参照された数 \\ \hline \hline
学校教育法 & 519 \\
私立学校教職員共済法 & 96 \\
私立学校法 & 58 \\
放送大学学園法 & 51 \\
教育職員免許法 & 31 \\
博物館法 & 31 \\
学校給食法 & 30 \\
社会教育法 & 29 \\
教育公務員特例法 & 25 \\
地方教育行政法 & 23 \\
学校教育法施行規則 & 23 \\
日本私立学校振興・共済事業団法 & 23 \\
\shortstack{公立学校の学校医、学校歯科医及び \\ 学校薬剤師の公務災害補償に関する法律} & 23 \\ \hline
  \end{tabular}
\end{table}

教育分野では、学校教育法が凄く大きなウェイトを占めている。学校教育法、学校教育法施行令、学校教育法施行規則 の教育法三法シリーズで、教育に関わることは幼稚園から大学まで大抵のことは決められているそうだ。学校教育法は519個も参照されている。よく話題に登る教育基本法は理念法のようで、実際は16個の法律しか参照していない。

面白いのは私立学校教職員共済法とか放送大学学園法が結構参照されている。
私立学校教職員共済法は私立学校の教職員の福利厚生に当たるもので、それらは関係する法律が多いんだなと思った。
放送大学学園法はなんでそんなに参照が多いのかよくわからなかった。

\section{憲法分野}

\includegraphics[width=9cm,clip]{img/kenpou-crop2.pdf}

憲法分野の法律の数は\textbf{220}本、参照数は\textbf{231}個である。

まず参照される法律の順に主要な法律を列挙する。

\begin{table}[htb]
  \begin{tabular}{|l|l|}  \hline
法律名 & 参照された数 \\ \hline \hline
日本国憲法 & 52 \\
沖縄の復帰に伴う特別措置に関する法律 & 51 \\
排他的経済水域及び大陸棚に関する法律 & 16 \\
国家賠償法 & 14 \\
\shortstack{連合国占領軍等の行為等による被害者等 \\ に対する給付金の支給に関する法律} & 12 \\
\shortstack{奄美群島の復帰に伴う法令の適用の \\ 暫定措置等に関する法律} & 11 \\
\shortstack{小笠原諸島の復帰に伴う法令の適用 \\  の暫定措置等に関する法律} & 10 \\
個人情報の保護に関する法律 & 10 \\ \hline
  \end{tabular}
\end{table}

憲法は国の中心となる法律だが意外とその参照先は多くない。学校教育法の様に500本近くの法律に参照されているということはなく、50本くらいの法律に参照されている。あくまで国の形を定めている法律だからだろう。
憲法分野の法律は、アメリカによる占領の跡が生々しく残っている。沖縄の復帰や小笠原諸島、奄美群島の復帰など、アメリカに占領されてしばらく統治下にあった場所が日本国に復帰する際に、結構色々な法律の整合性を取るためか、多くの法律に参照されている。
沖縄の復帰に関しての法律を参照する法律は51個と、日本国憲法が参照している法に並ぶ。

憲法分野では他に皇室典範なども属している。

憲法を参照している法律は少ないが、例えば警察法や裁判所法など警察や裁判所など国家の基本となる法から参照されており、グーグル的なページランクで言えば重要度が高いのかも知れない。

面白いのは宇宙基本法が憲法を参照しており、ある意味、宇宙も領土だと考えれば憲法に絡むのかなと思った\footnote{正確には「憲法の平和精神の理念を踏まえ」という形で参照している}。

\section{社会保険分野}

\includegraphics[width=8cm,clip]{img/社会保険-crop.pdf}

社会保険分野の法律の数は\textbf{430}本、参照数は\textbf{1332}個である。

まず参照される法律の順に主要な法律を列挙する。


\begin{table}[htb]
  \begin{tabular}{|l|l|}  \hline
法律名 & 参照された数 \\ \hline \hline
国民年金法 & 195 \\
厚生年金保険法 & 188 \\
健康保険法 & 181 \\
船員保険法 & 139 \\
国民健康保険法 & 101 \\
確定給付企業年金法 & 52 \\
確定拠出年金法 & 52 \\
社会保険診療報酬支払基金法 & 41 \\
健康保険法施行令 & 30 \\
日本年金機構法 & 28 \\
社会保険労務士法 & 23 \\ \hline
  \end{tabular}
\end{table}

見るからに複雑で絶対バグが入り込みそうな法律分野がある(笑)。社会保険分野だ。この分野を見ていくと国民年金の参照数が195個と多いが、それより目を引くのが相互参照の多さだ。各種保険、年金が相互に参照しあって訳がわからなくなっている。他の分野の様に中心となる法があってそれを取り巻く法律があるような構造というよりは、何が中心だかよくわからない構造になっている。多分この種の制度は利用者に不利にならないような制度設計をせねばならず、政治的な理由で頻繁に変わりまたその影響も大きいので、大量の特例や他の法律への参照が出来てどんどん複雑になっているのだろう。
プログラマーとしてはあんまり触りたくないコード(法律)だ。

法律間のつながりがどのくらい複雑なのか、前に見たシンプルな河川法分野と比較してみよう。

河川法分野の法律間のつながりの数を調べると、平均、1.18本(法律一本が平均で、1.18個参照されている)になる。
一方、社会保険分野を調査すると法律間の平均的なつながりの数は、3.098本と河川法分野に比べて約3倍複雑な仕組みになっている。

これから少子高齢化でますます各種年金制度は変更を余儀なくされるので更に複雑になっていくだろう。何か法律を一つ改正する度に、影響を受ける法律の精査を全てしなければならず厚生省の官僚は大変だろう。

\section{労働分野}

\includegraphics[width=9cm,clip]{img/労働-crop.pdf}

労働の法律の数は\textbf{575}本、参照数は\textbf{1731}個である。

まず参照される法律の順に主要な法律を列挙する。

\begin{table}[htb]
  \begin{tabular}{|l|l|}  \hline
法律名 & 参照された数 \\ \hline \hline
労働基準法 & 174 \\
労働者災害補償保険法 & 129 \\
雇用保険法 & 103 \\
職業安定法 & 84 \\
労働安全衛生法 & 84 \\
職業能力開発促進法 & 76 \\
船員職業安定法 & 74 \\
労働者派遣法 & 69 \\ \hline
  \end{tabular}
\end{table}

労働分野も、前にみた社会保険分野と同じくものすごく複雑に入り組んでいる。労働者の権利を守るという意味では、日頃からお世話になっているのだがこんなに複雑ではどうしてよいやらよくわからない。基本的には労働基準法が最も多くの参照を持つが、労働災害を担当する労働者災害補償保険法なども多くの参照を持っていて複雑に絡み合っている。大体法律の名前を見れば何している法律か分かるが、船員職業安定法だけなぜ個別の法案になっているのかよくわからなかった。

\clearpage


\section{国債分野}

\includegraphics[width=8cm,clip]{img/国債-crop.pdf}

国債に関する法律の数は\textbf{116}本、参照数は\textbf{144}個である。

まず参照される法律の順に主要な法律を列挙する。

\begin{table}[htb]
  \begin{tabular}{|l|l|}  \hline
法律名 & 参照された数 \\ \hline \hline
国債規則 & 34 \\
国債の発行等に関する省令 & 18 \\
中央省庁等改革関連政令 & 16 \\
日本銀行国債事務取扱規程 & 13 \\
政府資金調達事務取扱規則 & 10 \\
外貨公債の発行に関する法律 & 10 \\
\shortstack{昭和五十九年度の財政運営に必要な財源の\\確保を図るための特別措置等に関する法律} & 9 \\
\shortstack{東日本大震災からの復興のための\\施策を実施するために必要な財源\\の確保に関する特別措置法} & 9 \\
\shortstack{湾岸地域における平和回復活動を支援するため\\平成二年度において緊急に講ずべき\\財政上の措置に必要な財源の確保に係る\\臨時措置に関する法律} & 5 \\
\shortstack{電子情報処理組織を使用して処理する場合における\\国債の登録手続の特例に関する省令} & 5  \\ \hline
  \end{tabular}
\end{table}

国債に関する法律を可視化した。縦に長く今まで見てきた法律とは形が異なっており、面白いので取り上げる。中心になる法律は、大正時代に出来た国債の発行に関する規則(国債規則)だ。しかし、それよりも面白いのは、毎年発行している特例公債法(赤字国債発行法)である。湾岸戦争とか東日本大震災とか何かにつけて国債を発行しており、例えば盆暮れと無く実家にお金の無心に来るだめ息子みたいな風情である。湾岸戦争に必要とか、東日本大震災の復興に必要とか必要になる理由は否定し難いのが腹立たしい。

\chapter{終わりに}

法律の構造を情報技術を使って分析し可視化した。今後は、以下のことをやっていきたい。

\begin{itemize}
\item webなどを用いて、インタラクティブに法律間の構造を可視化するシステムを作る
\item  国別の法律を可視化する
\item 1945年、1955年、1965年...と年代ごとの法律の関連を可視化する(法律がどのように進化していったか分かる)。
\item 条文へのリンクを作る
\item どうしてこの法律とこの法律が繋がっているのか一発で分かるようにする
\item 判例でどの法律がどの程度使われているかも合わせて表示する
\end{itemize}

法律は今まで極めてアナログなものだと思われてきた。しかし、情報技術の発展によって法律も他の物と同じようにデータ分析の対象になってきている。私はこの流れを進めたいと思っている。

また、法律や行政の可視化に興味がある人は気軽に連絡をくれると嬉しい(連絡先は奥付を参照)。

\chapter{追記 可視化法学の作り方}

需要があるかどうかよくわからないが、可視化法学の作り方を追記しておく。可視化法学は次のように作った。


\begin{enumerate}
 \item 総務省法令データから、法令データを取得する
 \item 手元でリンク構造を抜き出す
 \item Gephiというソフトにデータを読み込ませグラフのレイアウトを決める
\end{enumerate}

というステップで可視化法学は作られた。

\section{法令データを取得する}

法令データの取得はこちらのサイトを利用した。

\textbf{総務省法令データ提供システム}

\url{http://law.e-gov.go.jp/cgi-bin/idxsearch.cgi}

ただ手軽には取れないので下のライブラリを利用した。

\url{https://github.com/riywo/law.e-gov.go.jp}

こちらの\textbf{download.rb}を利用して法令データを取得した。\footnote{法令データ自体は結構昔に取っていて、2013年に取得した。現在でもうまくいくかどうかは不明だ。}

国の機関なので大丈夫だとは思うが、あまり頻繁にデータを取得すると、サイバー攻撃と間違えられるので、ゆっくり取得してほしい。

\section{法律間のつながりを求める}

上記のライブラリを利用してデータを取得したら、次に法律間の関連を取得する。各法律には、aタグとしてリンク構造が埋め込まれている。例えば、救急救命士法 \footnote{\url{http://law.e-gov.go.jp/htmldata/H03/H03HO036.html}} には、刑法へのリンクがあり、次のようになっている。
参照は、htmlのaタグで、\verb|<|a herf="\url{http://law.e-gov.go.jp/cgi-bin/idxrefer.cgi?H_FILE=明四〇法四五&REF_NAME=刑法}"\verb|>|刑法\verb|<|/a\verb|>|をURLエンコードしたものになっており、それが刑法への参照になっている。

つまり法律内の文言でAタグを取得していけばREF\_NAMEの項目に各法律への参照を書いているので、そちらを利用して法律間の参照を取得していく。
私は プログラミング言語の \textbf{python} とデータストレージに \textbf{mongodb} を使い参照を取得するプログラムを作った。
実際の参照の取得に関しては少し時間がかかり、2日ほどプログラムを動かしたことを覚えている。

\section{Gephiで可視化する}

\begin{wrapfigure}{L}{70mm}
 \begin{center}
 \includegraphics[width=65mm,clip]{img/gephi.pdf}
 \caption{gephiスクリーンショット}
 \end{center}
\end{wrapfigure}

上の作業で、法律間の関連 「刑法」 ← 「救急救命士法」と言った参照構造が取得できたのでそちらの可視化を行う。

可視化はGephi\footnote{\url{http://oss.infoscience.co.jp/gephi/gephi.org/}}で行った。

YifanHu (胡一凡) という人の作ったアルゴリズムを使うときれいに表示できるようだ。

\newpage

\vspace*{\stretch{1}}
\begin{flushright}
\begin{minipage}{0.7\hsize}
\begin{description}[labelindent=1em ,labelwidth=0cm, labelsep*=1em, leftmargin =!, style = standard]% 
  \item{著者:} 芝尾幸一郎(へその緒)
  \item{twitter:} \verb|@lawvis|
  \item{mail:} \verb|shibacow@gmail.com|
  \item{発行:} 2017年04月09日
  \item{印刷:} トム出版 
\end{description}
著者として最善を尽くしていますが、この冊子を参考にして行われた運用結果に関して著者は一切の責任を負いかねますので、予めご了承ください。
\end{minipage}
\end{flushright}


%\newpage
%
%\thispagestyle{empty}
% %←白紙ページの為に全角文字列が入ってます。
%
%\newpage
%
%\thispagestyle{empty}
%
%\includegraphics[width=11cm,clip]{img/back.png}

\end{document}
